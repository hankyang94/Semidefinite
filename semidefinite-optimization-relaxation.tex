% Options for packages loaded elsewhere
\PassOptionsToPackage{unicode}{hyperref}
\PassOptionsToPackage{hyphens}{url}
%
\documentclass[
]{book}
\usepackage{amsmath,amssymb}
\usepackage{lmodern}
\usepackage{iftex}
\ifPDFTeX
  \usepackage[T1]{fontenc}
  \usepackage[utf8]{inputenc}
  \usepackage{textcomp} % provide euro and other symbols
\else % if luatex or xetex
  \usepackage{unicode-math}
  \defaultfontfeatures{Scale=MatchLowercase}
  \defaultfontfeatures[\rmfamily]{Ligatures=TeX,Scale=1}
\fi
% Use upquote if available, for straight quotes in verbatim environments
\IfFileExists{upquote.sty}{\usepackage{upquote}}{}
\IfFileExists{microtype.sty}{% use microtype if available
  \usepackage[]{microtype}
  \UseMicrotypeSet[protrusion]{basicmath} % disable protrusion for tt fonts
}{}
\makeatletter
\@ifundefined{KOMAClassName}{% if non-KOMA class
  \IfFileExists{parskip.sty}{%
    \usepackage{parskip}
  }{% else
    \setlength{\parindent}{0pt}
    \setlength{\parskip}{6pt plus 2pt minus 1pt}}
}{% if KOMA class
  \KOMAoptions{parskip=half}}
\makeatother
\usepackage{xcolor}
\usepackage{longtable,booktabs,array}
\usepackage{calc} % for calculating minipage widths
% Correct order of tables after \paragraph or \subparagraph
\usepackage{etoolbox}
\makeatletter
\patchcmd\longtable{\par}{\if@noskipsec\mbox{}\fi\par}{}{}
\makeatother
% Allow footnotes in longtable head/foot
\IfFileExists{footnotehyper.sty}{\usepackage{footnotehyper}}{\usepackage{footnote}}
\makesavenoteenv{longtable}
\usepackage{graphicx}
\makeatletter
\def\maxwidth{\ifdim\Gin@nat@width>\linewidth\linewidth\else\Gin@nat@width\fi}
\def\maxheight{\ifdim\Gin@nat@height>\textheight\textheight\else\Gin@nat@height\fi}
\makeatother
% Scale images if necessary, so that they will not overflow the page
% margins by default, and it is still possible to overwrite the defaults
% using explicit options in \includegraphics[width, height, ...]{}
\setkeys{Gin}{width=\maxwidth,height=\maxheight,keepaspectratio}
% Set default figure placement to htbp
\makeatletter
\def\fps@figure{htbp}
\makeatother
\setlength{\emergencystretch}{3em} % prevent overfull lines
\providecommand{\tightlist}{%
  \setlength{\itemsep}{0pt}\setlength{\parskip}{0pt}}
\setcounter{secnumdepth}{5}
\usepackage{booktabs}
\usepackage{amsthm}
\usepackage{amsmath}
\usepackage[pagebackref=true,breaklinks=true,letterpaper=true,colorlinks,bookmarks=true]{hyperref}
\usepackage{tcolorbox}
\usepackage{color}
\usepackage{framed}
\setlength{\fboxsep}{.8em}
\makeatletter
\def\thm@space@setup{%
  \thm@preskip=8pt plus 2pt minus 4pt
  \thm@postskip=\thm@preskip
}
\makeatother

% \newcommand{\Real}[1]{\mathbb{R}^{#1}}
% \newcommand{\sym}[1]{\mathbb{S}^{#1}}
% \newcommand{\psd}[1]{\sym{#1}_{+}}
% \newcommand{\pd}[1]{\sym{#1}_{++}}
% \newcommand{\inprod}[2]{\langle #1, #2 \rangle}
% \newcommand{\linprod}[2]{\left\langle #1, #2 \right\rangle}
% \newcommand{\trace}{\mathrm{tr}}
% \newcommand{\tran}{^\top}
% % \newcommand{\det}{\mathrm{det}}
% \newcommand{\rank}{\mathrm{rank}}
% \newcommand{\diag}{\mathrm{diag}}

\newtcolorbox{examplebox}{
  colback=green,
  colframe=orange,
  coltext=black,
  boxsep=5pt,
  arc=4pt}

\newtcolorbox{theorembox}{
  colback=green,
  colframe=green,
  coltext=black,
  boxsep=5pt,
  arc=4pt}

\newtcolorbox{definitionbox}{
colback=white,
colframe=green,
coltext=black,
boxsep=5pt,
arc=4pt}
\ifLuaTeX
  \usepackage{selnolig}  % disable illegal ligatures
\fi
\usepackage[]{natbib}
\bibliographystyle{apalike}
\IfFileExists{bookmark.sty}{\usepackage{bookmark}}{\usepackage{hyperref}}
\IfFileExists{xurl.sty}{\usepackage{xurl}}{} % add URL line breaks if available
\urlstyle{same} % disable monospaced font for URLs
\hypersetup{
  pdftitle={Semidefinite Optimization and Relaxation},
  pdfauthor={Heng Yang},
  hidelinks,
  pdfcreator={LaTeX via pandoc}}

\title{Semidefinite Optimization and Relaxation}
\author{Heng Yang}
\date{2024-01-19}

\usepackage{amsthm}
\newtheorem{theorem}{Theorem}[chapter]
\newtheorem{lemma}{Lemma}[chapter]
\newtheorem{corollary}{Corollary}[chapter]
\newtheorem{proposition}{Proposition}[chapter]
\newtheorem{conjecture}{Conjecture}[chapter]
\theoremstyle{definition}
\newtheorem{definition}{Definition}[chapter]
\theoremstyle{definition}
\newtheorem{example}{Example}[chapter]
\theoremstyle{definition}
\newtheorem{exercise}{Exercise}[chapter]
\theoremstyle{definition}
\newtheorem{hypothesis}{Hypothesis}[chapter]
\theoremstyle{remark}
\newtheorem*{remark}{Remark}
\newtheorem*{solution}{Solution}
\begin{document}
\maketitle

{
\setcounter{tocdepth}{1}
\tableofcontents
}
\newcommand{\Real}[1]{\mathbb{R}^{#1}}
\newcommand{\sym}[1]{\mathbb{S}^{#1}}
\newcommand{\psd}[1]{\sym{#1}_{+}}
\newcommand{\pd}[1]{\sym{#1}_{++}}
\newcommand{\inprod}[2]{\langle #1, #2 \rangle}
\newcommand{\linprod}[2]{\left\langle #1, #2 \right\rangle}
\newcommand{\trace}{\mathrm{tr}}
\newcommand{\tran}{^\top}

\newcommand{\rank}{\mathrm{rank}}
\newcommand{\diag}{\mathrm{diag}}
\newcommand{\Diag}{\mathrm{Diag}}
\newcommand{\BlkDiag}{\mathrm{BlkDiag}}
\newcommand{\vectorize}{\mathrm{vec}}
\newcommand{\svec}{\mathrm{svec}}
\newcommand{\mat}{\mathrm{mat}}
\newcommand{\smat}{\mathrm{smat}}
\newcommand{\norm}[1]{\Vert #1 \Vert}
\newcommand{\lnorm}[1]{\left\Vert #1 \right\Vert}
\newcommand{\pnorm}[2]{\Vert #1 \Vert_{#2}}
\newcommand{\Fnorm}[1]{\Vert #1 \Vert_\mathrm{F}}
\newcommand{\conv}{\mathrm{conv}}
\newcommand{\cone}{\mathrm{cone}}
\newcommand{\interior}{\mathrm{int}}
\newcommand{\poly}[1]{\mathbb{R}[#1]}
\newcommand{\SOd}{\mathrm{SO}(d)}
\newcommand{\SOthree}{\mathrm{SO}(3)}
\newcommand{\usphere}{\mathcal{S}}
\newcommand{\bmath}[1]{\boldsymbol{#1}}
\newcommand{\lbrkt}{[\![}
\newcommand{\rbrkt}{]\!]}
\newcommand{\brkt}[1]{\lbrkt #1 \rbrkt}
\newcommand{\cbrace}[1]{\{ #1 \}}
\newcommand{\lcbrace}[1]{ \left\{ #1 \right\} }

\hypertarget{preface}{%
\chapter*{Preface}\label{preface}}
\addcontentsline{toc}{chapter}{Preface}

This is the textbook for Harvard ENG-SCI 257: Semidefinite Optimization and Relaxation. Information about the offerings of the class is listed below.

\hypertarget{spring}{%
\subsubsection*{2024 Spring}\label{spring}}
\addcontentsline{toc}{subsubsection}{2024 Spring}

\textbf{Time}: Mon/Wed 2:15 - 3:30pm

\textbf{Location}: Science and Engineering Complex, 1.413

\textbf{Instructor}: \href{https://hankyang.seas.harvard.edu/}{Heng Yang}

\textbf{Teaching Fellow}: \href{https://safwanhossain.github.io/}{Safwan Hossain}

\href{https://docs.google.com/document/d/1H6Wqht_PVw_n8Jl0kXN3HjZfHkeZJYqYWT4ayxvqRlU/edit?usp=sharing}{\textbf{Syllabus}}

\hypertarget{acknowledgment}{%
\subsubsection*{Acknowledgment}\label{acknowledgment}}
\addcontentsline{toc}{subsubsection}{Acknowledgment}

\hypertarget{notation}{%
\chapter*{Notation}\label{notation}}
\addcontentsline{toc}{chapter}{Notation}

We will use the following standard notation throughout this book.

\textbf{Basics}

\begin{longtable}[]{@{}
  >{\raggedright\arraybackslash}p{(\columnwidth - 2\tabcolsep) * \real{0.7083}}
  >{\raggedright\arraybackslash}p{(\columnwidth - 2\tabcolsep) * \real{0.2917}}@{}}
\toprule()
\endhead
\(\mathbb{R}^{}\) & real numbers \\
\(\mathbb{R}^{}_{+}\) & nonnegative real \\
\(\mathbb{R}^{}_{++}\) & positive real \\
\(\mathbb{Z}\) & integers \\
\(\mathbb{N}\) & nonnegative integers \\
\(\mathbb{R}^{n}\) & \(n\)-D column vector \\
\(\mathbb{R}^{n}_{+}\) & nonnegative orthant \\
\(\mathbb{R}^{n}_{++}\) & positive orthant \\
\(e_i\) & standard basic vector \\
\(\Delta_n := \{x \in \mathbb{R}^n_{+} \mid \sum x_i = 1 \}\) & standard simplex \\
\bottomrule()
\end{longtable}

\textbf{Matrices}

\begin{longtable}[]{@{}
  >{\raggedright\arraybackslash}p{(\columnwidth - 2\tabcolsep) * \real{0.7083}}
  >{\raggedright\arraybackslash}p{(\columnwidth - 2\tabcolsep) * \real{0.2917}}@{}}
\toprule()
\endhead
\(\mathbb{R}^{m \times n}\) & \(m \times n\) real matrices \\
\(\mathbb{S}^{n}\) & \(n\times n\) symmetric matrices \\
\(\mathbb{S}^{n}_{+}\) & \(n\times n\) positive semidefinite matrices \\
\(\mathbb{S}^{n}_{++}\) & \(n\times n\) positive definite matrices \\
\(\langle A, B \rangle\) or \(\bullet\) & inner product in \(\mathbb{R}^{m \times n}\) \\
\(\mathrm{tr}(A)\) & trace of \(A \in \mathbb{R}^{n \times n}\) \\
\(A^\top\) & matrix transpose \\
\(\det(A)\) & matrix determinant \\
\(\mathrm{rank}(A)\) & rank of a matrix \\
\(\mathrm{diag}(A)\) & diagonal of a matrix \(A\) as a vector \\
\(\mathrm{Diag}(a)\) & turning a vector into a diagonal matrix \\
\(\mathrm{BlkDiag}(A,B,\dots)\) & block diagonal matrix with blocks \(A,B,\dots\) \\
\(\succeq 0\) and \(\preceq 0\) & positive / negative semidefinite \\
\(\succ 0\) and \(\prec 0\) & positive / negative definite \\
\(\lambda_{\max}\) and \(\lambda_{\min}\) & maximum / minimum eigenvalue \\
\(\sigma_{\max}\) and \(\sigma_{\min}\) & maximum / minimum singular value \\
\(\mathrm{vec}(A)\) & vectorization of \(A \in \mathbb{R}^{m \times n}\) \\
\(\mathrm{svec}(A)\) & symmetric vectorization of \(A \in \mathbb{S}^{n}\) \\
\(\Vert A \Vert_\mathrm{F}\) & Frobenius norm \\
\bottomrule()
\end{longtable}

\textbf{Geometry}

\begin{longtable}[]{@{}ll@{}}
\toprule()
\endhead
\(\Vert a \Vert_{p}\) & \(p\)-norm \\
\(\Vert a \Vert\) & \(2\)-norm \\
\(B(o,r)\) & ball with center \(o\) and radius \(r\) \\
\(\mathrm{conv}(S)\) & convex hull of set \(S\) \\
\(\mathrm{cone}(S)\) & conical hull of set \(S\) \\
\(\mathrm{int}(S)\) & interior of set \(S\) \\
\(\partial S\) & boundary of set \(S\) \\
\(P^\circ\) & polar dual of convex body \\
\(\mathrm{SO}(d)\) & special orthogonal group of dimension \(d\) \\
\(\mathcal{S}^{d-1}\) & unit sphere in \(\mathbb{R}^{d}\) \\
\bottomrule()
\end{longtable}

\textbf{Optimization}

\begin{longtable}[]{@{}ll@{}}
\toprule()
\endhead
KKT & Karush--Kuhn--Tucker \\
LP & linear program \\
QP & quadratic program \\
SOCP & second-order cone program \\
SDP & semidefinite program \\
\bottomrule()
\end{longtable}

\textbf{Algebra}

\begin{longtable}[]{@{}ll@{}}
\toprule()
\endhead
\(\mathbb{R}[x]\) & polynomial ring in \(x\) with real coefficients \\
\(\deg\) & degree of a monomial / polynomial \\
\(\mathbb{R}[x]_d\) & polynomials in \(x\) of degree up to \(d\) \\
\([x]_d\) & vector of monomials of degree up to \(d\) \\
\([\![x ]\!]_d\) & vector of monomials of degree \(d\) \\
\bottomrule()
\end{longtable}

\hypertarget{background}{%
\chapter{Mathematical Background}\label{background}}

\begin{itemize}
\tightlist
\item
  Convexity
\item
  Convex Optimization
\item
  Convex Geometry
\item
  Linear Programming
\end{itemize}

\hypertarget{background:convexity}{%
\section{Convexity}\label{background:convexity}}

A very important notion in modern optimization is that of \emph{convexity}. To a large extent, an optimization problem is ``easy'' if it is convex, and ``difficult'' when convexity is lost, i.e., \emph{nonconvex}. We give a basic review of convexity here and refer the reader to \citep{rockafellar70-convexanalysis}, \citep{boyd04book-convex}, and \citep{bertsekas03book-convex} for comprehensive treatments.

We will work on a finite-dimensional real vector space, which we will identify with \(\mathbb{R}^{n}\).

\begin{definition}[Convex Set]
\protect\hypertarget{def:ConvexSet}{}\label{def:ConvexSet}A set \(S\) is convex if \(x_1,x_2 \in S\) implies \(\lambda x_1 + (1-\lambda) x_2 \in S\) for any \(\lambda \in [0,1]\). In other words, if \(x_1,x_2 \in S\), then the line segment connecting \(x_1\) and \(x_2\) lies inside \(S\).
\end{definition}

Conversely, a set \(S\) is nonconvex if Definition \ref{def:ConvexSet} does not hold.

A \emph{hyperplane} is a common convex set defined as
\[
P = \{  x \in \mathbb{R}^{n} \mid \langle c, x \rangle = d  \}
\]
for some \(c \in \mathbb{R}^{n}\) and scalar \(d\). A \emph{halfspace} is a convex set defined as
\[
H = \{  x \in \mathbb{R}^{n} \mid \langle c, x \rangle \geq d  \}.
\]

An important property of a convex set is that we can \emph{certify} when a point is not in the set. This is usually done via a separation theorem.

\begin{theorem}[Separation Theorem]
\protect\hypertarget{thm:SeparationTheorem}{}\label{thm:SeparationTheorem}Let \(S_1,S_2\) be two convex sets in \(\mathbb{R}^{n}\) and \(S_1 \cap S_2 = \emptyset\), then there exists a hyperplane that separates \(S_1\) and \(S_2\), i.e., there exists \(c\) and \(d\) such that
\begin{equation}
\begin{split}
\langle c, x \rangle \geq d, &  \forall x \in S_1,\\
\langle c, x \rangle \leq d, & \forall x \in S_2.
\end{split}
\label{eq:separation}
\end{equation}
Further, if \(S_1\) is compact (i.e., closed and bounded) and \(S_2\) is closed, then the separation is strict, i.e., the inequalities in \eqref{eq:separation} are strict.
\end{theorem}

The strict separation theorem is used typically when \(S_1\) is a single point (hence compact).

The intersection of convex sets is always convex (try to prove this).

\hypertarget{background:convex:geometry}{%
\section{Convex Geometry}\label{background:convex:geometry}}

\begin{definition}[Extreme Point]
\protect\hypertarget{def:ExtremePoint}{}\label{def:ExtremePoint}Given a convex set, a point \(x \in S\) is \emph{extreme} if
\[
\forall x_1, x_2 \in S, \exists \lambda \in (0,1) \text { such that } x = \lambda x_1 + (1-\lambda) x_2 \Longrightarrow x = x_1 = x_2.
\]
In other words, any line segment that contains \(x\) and lies inside \(S\) must be \(x\) itself.
\end{definition}

\hypertarget{sdp}{%
\chapter{Semidefinite Optimization}\label{sdp}}

\begin{itemize}
\tightlist
\item
  Positive Semidefinite Matrices
\item
  Spectrahedra
\end{itemize}

  \bibliography{book.bib,packages.bib}

\end{document}
